\documentclass[a4paper, 14pt]{extarticle}
\usepackage[russian]{babel}
\usepackage[T1]{fontenc}
\usepackage{fontspec}
\usepackage{indentfirst}
\usepackage{enumitem}
\usepackage[
  left=20mm,
  right=10mm,
  top=20mm,
  bottom=20mm
]{geometry}
\usepackage{parskip}
\usepackage{titlesec}
\usepackage{xurl}
\usepackage{hyperref}
\usepackage{longtable}
\usepackage{array}
\usepackage{float}
\usepackage{graphicx}
\usepackage{diagbox}
\usepackage[
  figurename=Рисунок,
  labelsep=endash,
]{caption}
\usepackage{mdframed}

\newcolumntype{L}[1]{>{\raggedright\let\newline\\\arraybackslash\hspace{0pt}}m{#1}}
\newcolumntype{C}[1]{>{\centering\let\newline\\\arraybackslash\hspace{0pt}}m{#1}}
\newcolumntype{R}[1]{>{\raggedleft\let\newline\\\arraybackslash\hspace{0pt}}m{#1}}

\newmdenv[
  linewidth=2pt,
  align=center,
  topline=false,
  bottomline=false,
  rightline=false,
  skipabove=\topsep,
  skipbelow=\topsep,
]{siderules}

\hypersetup{
  colorlinks=true,
  linkcolor=black,
  filecolor=blue,
  urlcolor=blue,
}

\renewcommand*{\labelitemi}{---}
\linespread{1.5}
\setmainfont{PT Astra Serif}

\renewcommand{\baselinestretch}{1.5}
\setlength{\parindent}{1.25cm}
\setlength{\parskip}{6pt}

\setlength{\parindent}{1.25cm}
\setlist[itemize]{itemsep=0em,topsep=0em,parsep=0em,partopsep=0em,leftmargin=1.55\parindent}
\setlist[enumerate]{itemsep=0em,topsep=0em,parsep=0em,partopsep=0em,leftmargin=1.55\parindent}

\renewcommand{\thesection}{\arabic{section}.}
\renewcommand{\thesubsection}{\thesection\arabic{subsection}.}
\renewcommand{\thesubsubsection}{\thesubsection\arabic{subsubsection}.}

\titleformat{\section}
{\normalfont\bfseries}{\thesection}{0.5em}{}

\begin{document}

\begin{flushright}
  \textit{Швалов Даниил К33211}
\end{flushright}

\begin{center}
  \bfseries
  Домашнее задание №1

  <<Прокрастинация>>

  Задание №2
\end{center}

Я решил проанализировать то, в какой ситуации я оказался. Для этого я выписал
задачи, которые я хотел бы или мне нужно было бы выполнить в ближайшее время:
\begin{itemize}
  \item через два дня дедлайн выданной сегодня лабораторной работы;
  \item сегодня концерт любимой группы, на который я давно мечтал попасть;
  \item грязная посуда, оставшаяся с утра;
  \item новый эпизод моего любимого сериала.
\end{itemize}

Я думаю самым лучшем решением этой задачи будет описать то, как я думаю в
подобных ситуациях, а потом проанализировать ход моих мыслей. Поэтому сначала я
опишу сам процесс моих размышлений, а после проведу анализ.

Я только пришел домой, скорее всего я немного уставший и хочу есть. Но вот
незадача, чистой посуды нет, потому что вся посуда ждет, пока я ее помою. В
такой ситуации я обычно поступаю следующим образом. Сначала мою самый
необходимый минимум, чтобы можно было поесть. Накладываю себе еду, ставлю
разогреваться. А пока греется, не теряю время и начинаю мыть оставшуюся посуду.
Понятно, что за две-три минуты я не вымою всю гору посуды, но явно смогу
немного поправить свое положение. Как еда согрелась, я конечно же иду есть.
Обычно после нормальной еды я также пью чай. Конечно же, нужно дождаться, пока
чайник вскипятит воду, это тоже занимает пару минут. И эти несколько минут
можно с радостью потратить на немытую посуду. Если в конце концов посуда не
была домыта, то после чайной церемонии придется еще немного потрудиться, чтобы
домыть посуду. Иначе эта история с посудой может повториться уже завтра.

Отлично, я сыт, посуда вымыта. Теперь нужно решить, чем я займу остаток дня.
Передо мной стоит непростой выбор: пойти на концерт любимой группы, сесть
делать лабораторную работу или посмотреть новый эпизод моего любимого сериала.
Сериал я всегда успею посмотреть позже, а если кто-то из друзей решит
заспойлерить, мне все равно. Ведь интересен не сам исход, а то, что к нему
привело. Задание по лабораторной работе, на которую преподаватель дал всего два
дня, либо очень простая и делается за час, либо сложная и за два дня не
делается. В любом случае, если это не конец семестра, то ничего страшного не
произойдет, если я сдам лабораторную работу с потерей баллов. А может вообще
успею ее сделать завтра. Я все равно вряд ли бы сегодня сел за нее. Я и так
устал уже от этого предмета, несколько пар провел за ним, а тут и еще и
лабораторка очередная. А вот концерт любимая группа дает редко, чуть ли не раз
в год. Сомневаюсь, что в ближайшие несколько месяцев смогу сходить на их
концерт.

Итого, я расставил следующий приоритет выполнения задач:
\begin{enumerate}
  \item грязная посуда, оставшаяся с утра;
  \item сегодня концерт любимой группы, на который я давно мечтал попасть;
  \item через два дня дедлайн задания для лабораторной работы;
  \item новый эпизод моего любимого сериала.
\end{enumerate}

Другими словами, с наибольшей вероятностью я отложу просмотр нового эпизода
сериала, а также новое задание по лабораторной работе. Если хватит времени, то
задание по лабораторной все же будет сделано в срок (все таки есть еще и
завтра).

Для лабораторной работы как триггер я могу выделить то, что скорее всего, раз
на нее отводится два дня, то задача не имеет ценности для меня как для
личности. Вряд ли она очень сложная или сильно скучная, раз ее так скоро
сдавать, но и также вряд ли и то, что она сильно полезная или очень интересная.

Все конечно зависит от случая. Может быть предмет мне сильно нравится и вместо
концерта я стану делать задание по лабораторной работе, а может я просто
концерты не очень люблю. Может быть такое, что сам предмет мне не очень
нравится, а задание будет интересным. Поэтому мне сложно выделить универсальный
способ борьбы с прокрастинацией в данном случае, поскольку от случая к случаю
все может быть по-разному.

\end{document}
