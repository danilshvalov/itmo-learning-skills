\documentclass[a4paper, 14pt]{extarticle}
\usepackage[russian]{babel}
\usepackage[T1]{fontenc}
\usepackage{fontspec}
\usepackage{indentfirst}
\usepackage{enumitem}
\usepackage[
  left=20mm,
  right=10mm,
  top=20mm,
  bottom=20mm
]{geometry}
\usepackage{parskip}
\usepackage{titlesec}
\usepackage{xurl}
\usepackage{hyperref}
\usepackage{longtable}
\usepackage{array}
\usepackage{float}
\usepackage{graphicx}
\usepackage{diagbox}
\usepackage[
  figurename=Рисунок,
  labelsep=endash,
]{caption}
\usepackage{mdframed}

\newcolumntype{L}[1]{>{\raggedright\let\newline\\\arraybackslash\hspace{0pt}}m{#1}}
\newcolumntype{C}[1]{>{\centering\let\newline\\\arraybackslash\hspace{0pt}}m{#1}}
\newcolumntype{R}[1]{>{\raggedleft\let\newline\\\arraybackslash\hspace{0pt}}m{#1}}

\newmdenv[
  linewidth=2pt,
  align=center,
  topline=false,
  bottomline=false,
  rightline=false,
  skipabove=\topsep,
  skipbelow=\topsep,
]{siderules}

\hypersetup{
  colorlinks=true,
  linkcolor=black,
  filecolor=blue,
  urlcolor=blue,
}

\renewcommand*{\labelitemi}{---}
\linespread{1.5}
\setmainfont{PT Astra Serif}

\renewcommand{\baselinestretch}{1.5}
\setlength{\parindent}{1.25cm}
\setlength{\parskip}{6pt}

\setlength{\parindent}{1.25cm}
\setlist[itemize]{itemsep=0em,topsep=0em,parsep=0em,partopsep=0em,leftmargin=1.55\parindent}
\setlist[enumerate]{itemsep=0em,topsep=0em,parsep=0em,partopsep=0em,leftmargin=1.55\parindent}

\renewcommand{\thesection}{\arabic{section}.}
\renewcommand{\thesubsection}{\thesection\arabic{subsection}.}
\renewcommand{\thesubsubsection}{\thesubsection\arabic{subsubsection}.}

\titleformat{\section}
{\normalfont\bfseries}{\thesection}{0.5em}{}

\begin{document}

\begin{flushright}
  \textit{Швалов Даниил К33211}
\end{flushright}

\begin{center}
  \bfseries
  Домашнее задание №2

  <<Мнемотехники>>
\end{center}

\section*{Метод Цицерона}

Метод Цицерона используется для запоминания списков дел или продуктов, речи для
конференции, материалов лекции и в целом для запоминания больших объемов
информации. Метод заключается в том, чтобы создать в воображении некое
пространство с опорными образами, а затем ассоциировать эти образы с тем, что
нужно запомнить.

В качестве образов стоит выбирать предметы, которые имеют четкое и постоянное
расположение в пространстве. Так, например, стол, шкаф, картина --- это хорошие
опорные образы, потому что они не меняют свое местоположение. В то же время
кружка или тарелка --- не самые подходящие опорные объекты, поскольку они часто
перемещаются в пространстве. При этом в качестве образов следует выбирать
достаточно крупные предметами. Также желательно, чтобы образы не повторялись
между собой.

Опорные образы стоит располагать в определенном направлении, например, по
часовой стрелке. Это нужно для того, чтобы выстраивалась определенная цепочка
информации, по которой можно было бы следовать, как бы идя по помещению.

\section*{Пиктограммы}

Для запоминания больших объемов информации также можно использовать метод
пиктограмм. Этот метод отлично подходит людям-визуалам, которые лучше всего
запоминают именно зрительные образы. Данный метод устроен следующим образом.

Допустим, есть текст, который необходимо запомнить. Для этого текста нужно
выделить ключевые тезисы и выписать их. Для каждого такого тезиса нужно
нарисовать простую картинку --- пиктограмму, с которой будет ассоциироваться
тезис. В качестве картинки может выступать что угодно: человечки, животные,
геометрические фигуры и т. п. После этого необходимо запомнить
последовательность, в которой идут эти картинки.

Пиктограммы могут выступать не только как метод запоминания информации, но и как
шпаргалки. Например, выступая с докладом, можно положить перед собой небольшой
листочек с пиктограммами. Если что-то вдруг вылетит из головы, пиктограммы
помогут вспомнить забытое.

\end{document}
