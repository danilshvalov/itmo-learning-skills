\documentclass[a4paper, 14pt]{extarticle}
\usepackage[russian]{babel}
\usepackage[T1]{fontenc}
\usepackage{fontspec}
\usepackage{indentfirst}
\usepackage{enumitem}
\usepackage[
  left=20mm,
  right=10mm,
  top=20mm,
  bottom=20mm
]{geometry}
\usepackage{parskip}
\usepackage{titlesec}
\usepackage{xurl}
\usepackage{hyperref}
\usepackage{longtable}
\usepackage{array}
\usepackage{float}
\usepackage{graphicx}
\usepackage{diagbox}
\usepackage[
  figurename=Рисунок,
  labelsep=endash,
]{caption}
\usepackage{mdframed}

\newcolumntype{L}[1]{>{\raggedright\let\newline\\\arraybackslash\hspace{0pt}}m{#1}}
\newcolumntype{C}[1]{>{\centering\let\newline\\\arraybackslash\hspace{0pt}}m{#1}}
\newcolumntype{R}[1]{>{\raggedleft\let\newline\\\arraybackslash\hspace{0pt}}m{#1}}

\newmdenv[
  linewidth=2pt,
  align=center,
  topline=false,
  bottomline=false,
  rightline=false,
  skipabove=\topsep,
  skipbelow=\topsep,
]{siderules}

\hypersetup{
  colorlinks=true,
  linkcolor=black,
  filecolor=blue,
  urlcolor=blue,
}

\renewcommand*{\labelitemi}{---}
\linespread{1.5}
\setmainfont{PT Astra Serif}

\renewcommand{\baselinestretch}{1.5}
\setlength{\parindent}{1.25cm}
\setlength{\parskip}{6pt}

\setlength{\parindent}{1.25cm}
\setlist[itemize]{itemsep=0em,topsep=0em,parsep=0em,partopsep=0em,leftmargin=1.55\parindent}
\setlist[enumerate]{itemsep=0em,topsep=0em,parsep=0em,partopsep=0em,leftmargin=1.55\parindent}

\renewcommand{\thesection}{\arabic{section}.}
\renewcommand{\thesubsection}{\thesection\arabic{subsection}.}
\renewcommand{\thesubsubsection}{\thesubsection\arabic{subsubsection}.}

\titleformat{\section}
{\normalfont\bfseries}{\thesection}{0.5em}{}

\begin{document}

\begin{flushright}
  \textit{Швалов Даниил К33211}
\end{flushright}

\begin{center}
  \bfseries
  Домашнее задание №3

  <<Техники эффективного чтения>>
\end{center}

В качестве темы, для которой необходимо найти источники, я выбрал тему
<<Принципы и методики здорового сна>>. Я считаю, что в эпоху цифровых
технологий, когда нас окружает куча устройств, способных отвлечь и вмешаться в
наш сон, это очень актуальная тема. Я сам постоянно сталкиваюсь с тем, что
не могу уснуть, потому что мне кто-то пишет или у меня появилось желание
посмотреть еще какое-нибудь видео или полистать мемчики и так до бесконечности.
Основная моя цель --- изучить принципы и методики здорового сна, чтобы спать
лучше, высыпаться чаще, быть здоровее и чувствовать себя бодрее.

В качестве источника информации я решил выбрать научную электронную библиотеку
elibrary --- российскую научную электронную библиотеку, интегрированную с
Российским индексом научного цитирования (РИНЦ). В итоге я нашел следующие источники по выбранной теме:
\begin{enumerate}
  \item Варганов, Д. Р. Особенности здорового сна / Д. Р. Варганов, А. С.
  Меркулова // Актуальные проблемы здоровьесбережения в современном обществе :
  сборник научный статей III-й Всероссийской научно-практической конференции,
  Курск, 07 октября 2021 года. – Курск: Юго-Западный государственный университет,
  2021. – С. 19-21. – EDN UKTQBN.

  \item Александров, Н. С. Сон как фактор здорового образа жизни / Н. С.
  Александров, Е. В. Иванова // Рефлексия. – 2023. – № 2. – С. 94-97. – EDN
  LOEJKD.

  \item Постнова, Н. Ю. Здоровый Сон и его влияние на современного человека / Н.
  Ю. Постнова // Инновационное развитие регионов: потенциал науки и современного
  образования : материалы V Национальной научно-практической конференции с
  международным участием, приуроченной ко Дню российской науки, Астрахань, 08–09
  февраля 2022 года. – Астрахань: Астраханский государственный
  архитектурно-строительный университет, 2022. – С. 288-291. – EDN KIIIPG.

  \item Головчан, Е. Н. Критерии здорового сна / Е. Н. Головчан // Образование.
  Наука. Производство : Сборник докладов XIV Международного молодежного форума,
  Белгород, 13–14 октября 2022 года. Том Часть 18. – Белгород: Белгородский
  государственный технологический университет им. В.Г. Шухова, 2022. – С. 165-169.
  – EDN CMGZWX.

  \item Даудов, Д. Р. Структура здорового сна человека / Д. Р. Даудов, Т. Р.
  Хамирзоев // Молодой ученый. – 2023. – № 4(451). – С. 101-102. – EDN DYKWAG.
\end{enumerate}


\end{document}
