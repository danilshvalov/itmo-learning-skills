\documentclass[a4paper, 14pt]{extarticle}
\usepackage[russian]{babel}
\usepackage[T1]{fontenc}
\usepackage{fontspec}
\usepackage{indentfirst}
\usepackage{enumitem}
\usepackage[
  left=20mm,
  right=10mm,
  top=20mm,
  bottom=20mm
]{geometry}
\usepackage{parskip}
\usepackage{titlesec}
\usepackage{xurl}
\usepackage{hyperref}
\usepackage{longtable}
\usepackage{array}
\usepackage{float}
\usepackage{graphicx}
\usepackage{diagbox}
\usepackage[
  figurename=Рисунок,
  labelsep=endash,
]{caption}
\usepackage{mdframed}
\usepackage{longtable}

\newcolumntype{L}[1]{>{\raggedright\let\newline\\\arraybackslash\hspace{0pt}}m{#1}}
\newcolumntype{C}[1]{>{\centering\let\newline\\\arraybackslash\hspace{0pt}}m{#1}}
\newcolumntype{R}[1]{>{\raggedleft\let\newline\\\arraybackslash\hspace{0pt}}m{#1}}

\newmdenv[
  linewidth=2pt,
  align=center,
  topline=false,
  bottomline=false,
  rightline=false,
  skipabove=\topsep,
  skipbelow=\topsep,
]{siderules}

\hypersetup{
  colorlinks=true,
  linkcolor=black,
  filecolor=blue,
  urlcolor=blue,
}

\renewcommand*{\labelitemi}{---}
\linespread{1.5}
\setmainfont{PT Astra Serif}

\renewcommand{\baselinestretch}{1.5}
\setlength{\parindent}{1.25cm}
\setlength{\parskip}{6pt}

\setlength{\parindent}{1.25cm}
\setlist[itemize]{itemsep=0em,topsep=0em,parsep=0em,partopsep=0em,leftmargin=1.55\parindent}
\setlist[enumerate]{itemsep=0em,topsep=0em,parsep=0em,partopsep=0em,leftmargin=1.55\parindent}

\renewcommand{\thesection}{\arabic{section}.}
\renewcommand{\thesubsection}{\thesection\arabic{subsection}.}
\renewcommand{\thesubsubsection}{\thesubsection\arabic{subsubsection}.}

\titleformat{\section}
{\normalfont\bfseries}{\thesection}{0.5em}{}

\begin{document}

\begin{flushright}
  \textit{Швалов Даниил К33211}
\end{flushright}

\begin{center}
  \bfseries
  Домашнее задание №4

  <<Работа с информацией>>
\end{center}

В качестве темы в прошлом домашнем задании я выбрал тему <<Принципы и методики
здорового сна>>. В качестве техники конспектирования я выбрал метод заметок
Корнелла. На мой взгляд, это самый удобный способ вести конспект для многих сфер
обучения. В следующей таблице представлены основные мысли и заметки того, что я
изучил в источниках, указанных в предыдущем домашнем задании.

\begin{longtable}{|C{0.25\textwidth}|L{0.7\textwidth}|}
  \hline
  \textbf{Основные мысли} & \multicolumn{1}{c|}{\textbf{Заметки}} \\
  \hline
  Последствия недостатка сна или некачественного сна
                          &
  Недостаток сна негативно влияет на производительность, на способность ясно
  мыслить, быстро реагировать, критически мыслить. Ухудшается способность
  концентрироваться, запоминать новую информацию, оказывается значительное
  влияние на физическую работоспособность: время реакции, координацию,
  выносливость. Повышается риск возникновения проблем с психическим здоровьем:
  раздражительность, депрессия, тревога и расстройства настроения. Появляется
  усталость, падает мотивированность к выполнению задач. Также увеличивается риск
  возникновения проблем со здоровьем: высокое кровяное давление, сердечная
  недостаточность, болезни почек, ожирение, сахарный диабет. При недостатке сна
  могут возникать эмоциональное перенапряжение, стрессы, усталость, изнеможение,
  головные боли, апатия, слабости в мышцах. Могут также возникать трудности с
  засыпанием или продолжением сна.
  \\
  \hline
  Способы улучшения количества и качества сна
                          &
  Для улучшения качества сна, следует руководствоваться следующими принципами.
  Во-первых, нужно придерживаться режима сна, т. е. ложиться спать и
  просыпаться в одно и то же время каждый день. Во-вторых, нужно поддерживать
  благоприятную для сна обстановку: низкая температура в помещении, комфортная
  влажность, отсутствие посторонних шумов, полная темнота, удобный матрас и
  подушки. Также стоит использовать различные методы релаксации перед сном, например,
  чтение или медитацию. В-третьих, перед сном следует ограничить использование
  гаджетов, потребление кофеина и алкоголя, количество физических нагрузок.

  Улучшить ситуацию также может короткий дневной сон. Он позволит нейтрализовать
  излишки кортизола, предупредит состояние подавленности, увеличенного давления,
  может улучшить память и процесс обучения. Также дневной сон способствует
  продлению работоспособности и заставляет организм бодрствовать до позднего
  вечера, несмотря на накопленную усталость.
  \\
  \hline
\end{longtable}

Сон играет ключевую роль в здоровье человека. Недостаток сна, а также
некачественный сон влияют на множество сфер нашей деятельности. Чтобы быть
здоровым, полным сил, чувствовать себя бодрым, оставаться мотивированным, хорошо
запоминать и усваивать материал, нужно уделять особое внимание сну.

\end{document}
